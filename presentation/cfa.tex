\begin{frame}[fragile]{Control Flow Analysis\footcite{shivers1991control}}
  \footnotesize
  \begin{columns}
    \begin{column}{0.75\textwidth}
  \begin{semiverbatim}\footnotesize
def error() = \{ \textcolor{red}{abort}; return 0 \}
def success() = return 0
def choice(b, f, g) = \{
  if (b) _ := \alert<4>{f}();
  else   _ := \alert<4>{g}();
  return 0
\}
choice(true, success, error);
choice(false, error, success);
\end{semiverbatim}\vspace{-1em}
\begin{itemize}
\item<4-> Whether \texttt{error} is called depends on parameters on different calls
\item<5-> We need to keep track of calling contexts to know what \texttt{f} and \texttt{g} are
\item<6-> With control flow analysis we figure out an abstract CFG while analyzing the program
\end{itemize}
\end{column}
\begin{column}{0.25\textwidth}
    \multiinclude[<+->][start=1,format=pdf,graphics={width=\textwidth}]{fig/cfg}
\end{column}
\end{columns}
\end{frame}

\begin{frame}[fragile]{Terminology}
  \begin{columns}[t]
    \begin{column}{.45\textwidth}
      \begin{block}{Flow sensitive analysis}
        Takes into consideration the order of statements (how the control flows)
      \end{block}
      \begin{example}[Flow Insensitive]
        \begin{lstlisting}[language=Java,mathescape]
          x := 3;
          y := x + 1; // $y \in \Set{4,43}$
          x := 42;
        \end{lstlisting}
      \end{example}
    \end{column}
    \begin{column}{.45\textwidth}
      \pause
      \begin{block}{Context sensitive analysis}
        Differentiates calls to the same function (the calling context)
      \end{block}
      \begin{example}[Context Insensitive]
        \begin{lstlisting}[language=Java,mathescape]
          x := add1(2); // $x \in \Set{3,4}$
          y := add1(3); // $y \in \Set{3,4}$
        \end{lstlisting}
      \end{example}
    \end{column}
  \end{columns}
\end{frame}
        
\begin{frame}{$k$-CFA\footcite{shivers1991control}}
  \begin{columns}
  \begin{column}{0.5\textwidth}
  \begin{itemize}[<+->]
  \item Duplicating functions in the control flow graph depending on
    how they are called would increase precision
  \item $k$-CFA: each abstract state contains a context with the last
    $k$ call sites
  \end{itemize}
  \end{column}
  
  \begin{column}{0.5\textwidth}
  \begin{exampleblock}{Example with $k=2$}
    \uncover<3->{\begin{align*}
                   \langle \ell_3: \textbf{baz}(7), \ldots, [\textbf{foo}_{\ell_1}, \textbf{bar}_{\ell_2}]\rangle \to\\
                   \langle \ell_4: \textbf{baz}_{entry}, \ldots, [\textbf{bar}_{\ell_2},\textbf{baz}_{\ell_3}] \rangle
                   \end{align*}}
  \end{exampleblock}
  \end{column}
  \end{columns}
\end{frame}

%%% Local Variables:
%%% mode: latex
%%% TeX-master: "main"
%%% TeX-engine: xetex
%%% End:
