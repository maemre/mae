% Unicode math symbols for XeLaTeX
\usepackage{mathrsfs}

% standard math packages
\usepackage{amssymb}
\usepackage{amsthm}
\usepackage{amsmath}
\usepackage{amstext}
% math tools for amsmath
\usepackage{mathtools}

\usepackage{proof}

% tikz & friends

\usepackage{galois}
\usepackage{tikz}
\usetikzlibrary{fit,calc,shapes,arrows.meta}
\usepackage[beamer]{hf-tikz}
\usepackage{tikz-cd}

\newcommand{\tikzmark}[1]{%
  \tikz[overlay,remember picture,baseline] \node [anchor=base] (#1) {};}

\newenvironment{tightcenter}{%
  \setlength\topsep{0pt}
  \setlength\parskip{0pt}
  \begin{center}
}{%
  \end{center}
}

% new colors
\definecolor{lightblue}{RGB}{217,220,253}
\definecolor{lightred}{RGB}{251,216,218}

\DeclareMathOperator{\pw}{\mathcal{P}} % powerset
\newcommand{\fset}[1]{\mathsf{#1}}
\newcommand{\nats}{\mathbb{N}}
\newcommand{\zahlen}{\mathbb{Z}}
\newcommand{\bools}{\mathbb{B}}
\newcommand{\Set}[1]{\left\{#1\right\}}
\newcommand{\kw}[1]{\textbf{#1}} % keywords
\newcommand{\true}{\kw{true}}
\newcommand{\false}{\kw{false}}
\newcommand{\sidenote}[1]{\hfill\quad \textsf{#1}}

\newcommand{\pitype}[1]{\Pi (#1).\,}
% \newcommand{\|}{\,\vert\,}
\newcommand{\todo}[1]{{\tiny \sffamily \textcolor{red}{TODO: #1}}}

% ceiling and floor symbols
\DeclarePairedDelimiter\ceil{\lceil}{\rceil}
\DeclarePairedDelimiter\floor{\lfloor}{\rfloor}

% big O notation
\DeclareMathOperator{\bigO}{O}

% fixed points
\DeclareMathOperator{\lfp}{lfp}

% print both years for bibliography
\renewbibmacro*{cite:labelyear+extrayear}{%
\iffieldundef{labelyear}
{}
{\printtext[bibhyperref]{%
\iffieldundef{origyear}{}{\printfield{origyear}\addslash}%   <--- added
\printfield{labelyear}%
\printfield{extrayear}}}}

\renewbibmacro*{date+extrayear}{%
\iffieldundef{year}
{}
{\printtext[parens]{%
\iffieldundef{origyear}{}{\printfield{origyear}\addslash}%  <--- added
\printdateextra}}}

% overlay an image
\def\Put(#1,#2)#3{\leavevmode\makebox(0,0){\put(#1,#2){#3}}}

% text over symbols nicely, requires amsmath for overset
\newcommand\textoverop[2]{\mathrel{\overset{\makebox[0pt]{\mbox{\normalfont\tiny\sffamily #1}}}{#2}}}

% special arrows
\newcommand\monarrow{\textoverop{mon}{\rightarrow}}

% theorems
\newtheorem{thm}{Theorem}

%%% Local Variables:
%%% mode: latex
%%% TeX-master: "main"
%%% TeX-engine: xetex
%%% End:
