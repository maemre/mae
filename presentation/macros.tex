% multimedia in beamer presentations
\usepackage{multimedia}

% Unicode math symbols for XeLaTeX
\usepackage{mathrsfs}

% standard math packages
\usepackage{amssymb}
\usepackage{amsthm}
\usepackage{amsmath}
\usepackage{amstext}
% math tools for amsmath
\usepackage{mathtools}

\usepackage{proof}

\usepackage{listings}
\usepackage{xcolor}

% tikz & friends

\usepackage{galois}
\usepackage{tikz}
\usetikzlibrary{fit,calc,shapes,arrows.meta,patterns,backgrounds}
\usetikzlibrary{cd}
\usepackage[beamer]{hf-tikz}

\newcommand{\tikzmark}[1]{%
  \tikz[overlay,remember picture,baseline] \node [anchor=base] (#1) {};}

\newenvironment{tightcenter}{%
  \setlength\topsep{0pt}
  \setlength\parskip{0pt}
  \begin{center}
}{%
  \end{center}
}

% new colors
\definecolor{lightblue}{RGB}{217,220,253}
\definecolor{lightred}{RGB}{251,216,218}

\DeclareMathOperator{\pw}{\mathcal{P}} % powerset
\newcommand{\fset}[1]{\mathsf{#1}}
\newcommand{\nats}{\mathbb{N}}
\newcommand{\zahlen}{\mathbb{Z}}
\newcommand{\bools}{\mathbb{B}}
\newcommand{\Set}[1]{\left\{#1\right\}}
\newcommand{\true}{\kw{true}}
\newcommand{\false}{\kw{false}}
\newcommand{\sidenote}[1]{\hfill\quad \textsf{#1}}

% named set
\newcommand{\ns}[1]{\mathit{#1}}
% function
\newcommand{\fn}[1]{\mathrm{#1}}
% "vector of"
\newcommand{\vo}[1]{\overrightarrow{#1}}
% "set of"
\newcommand{\setOf}[1]{\overline{#1}}
% syntactic tag
\newcommand{\sTag}[2]{\textsf{\textbf{#1}}\,#2}
% keyword
\newcommand{\kw}[1]{\texttt{#1}}
% usual suspects
\newcommand{\State}{\ns{State}}
\newcommand{\Value}{\ns{Value}}
\newcommand{\Stmt}{\ns{Stmt}}
\newcommand{\Env}{\ns{Env}}
\newcommand{\Store}{\ns{Store}}
\newcommand{\Kont}{\ns{Kont}}
% syntactic domains
\newcommand{\Exp}{\ns{Exp}}
\newcommand{\Var}{\ns{Var}}
% put a value to a pointer
\newcommand{\update}{\leftarrow}

% ebnf
\newcommand{\eDEF}{\,::=\;}
\newcommand{\eOR}{\;\vert\;}

\newcommand{\widen}{\nabla}

% \newcommand{\|}{\,\vert\,}
\newcommand{\todo}[1]{{\textcolor{red}{TODO: #1}}}

% ceiling and floor symbols
\DeclarePairedDelimiter\ceil{\lceil}{\rceil}
\DeclarePairedDelimiter\floor{\lfloor}{\rfloor}

% big O notation
\DeclareMathOperator{\bigO}{O}

% fixed points
\DeclareMathOperator{\lfp}{lfp}

% print both years for bibliography
\renewbibmacro*{cite:labelyear+extrayear}{%
\iffieldundef{labelyear}
{}
{\printtext[bibhyperref]{%
\iffieldundef{origyear}{}{\printfield{origyear}\addslash}%   <--- added
\printfield{labelyear}%
\printfield{extrayear}}}}

\renewbibmacro*{date+extrayear}{%
\iffieldundef{year}
{}
{\printtext[parens]{%
\iffieldundef{origyear}{}{\printfield{origyear}\addslash}%  <--- added
\printdateextra}}}

% overlay an image
\def\Put(#1,#2)#3{\leavevmode\makebox(0,0){\put(#1,#2){#3}}}

% text over symbols nicely, requires amsmath for overset
\newcommand\textoverop[2]{\mathrel{\overset{\makebox[0pt]{\mbox{\normalfont\tiny\sffamily #1}}}{#2}}}

% special arrows
\newcommand\monarrow{\textoverop{mon}{\rightarrow}}

% theorems
\newtheorem{thm}{Theorem}
\newtheorem{eg}{Example}

\newcommand{\abscolor}[1]{\textcolor{magenta}{#1}}
\newcommand{\concolor}[1]{\textcolor{blue}{#1}}
\newcommand{\abst}[1]{#1^{\#}}

\newcommand{\step}{\rightsquigarrow}

%%% Local Variables:
%%% mode: latex
%%% TeX-master: "main"
%%% TeX-engine: xetex
%%% End:
